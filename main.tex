\documentclass[zihao=-4,a4paper]{ctexart}

%===================   导言区    ================================================================
% 加载字体相关宏包,支持设置英文字体和中文字体
\usepackage{fontspec}       % fontspec 宏包用于设置 OpenType 或 TrueType 字体,适用于 XeLaTeX 或 LuaLaTeX
\usepackage{xeCJK}          % xeCJK 宏包用于支持中文字体设置和排版

% 设置中文字体和主字体
\newcommand{\zhongsong}{\CJKfontspec{华文中宋}}  % 定义 \zhongsong 命令,使用华文中宋字体用于中文部分
\setmainfont{Times New Roman}                    % 设置主字体为 Times New Roman,适用于英文部分

% 数学相关
\usepackage{amsmath}    % AMS LaTeX 数学公式增强(align, cases, split 等)
\usepackage{amsthm}     % 定理环境(theorem, lemma, corollary)
\usepackage{amsfonts}   % 数学字体(\mathbb, \mathfrak)
\usepackage{mathrsfs}   % 额外的花体数学符号(\mathscr)
\usepackage{bm}         % 加粗数学符号(\bm)
\newtheorem{example}{例}              % 创建一个新的定理环境,命名为 "example",并设置标题为 "例"(整体编号)
\newtheorem{theorem}{定理}            % 创建一个新的定理环境,命名为 "theorem",并设置标题为 "定理"(按章节编号)
\newtheorem{definition}{定义}         % 创建一个新的定理环境,命名为 "definition",并设置标题为 "定义"
\newtheorem{axiom}{公理}              % 创建一个新的定理环境,命名为 "axiom",并设置标题为 "公理"
\newtheorem{property}{性质}           % 创建一个新的定理环境,命名为 "property",并设置标题为 "性质"
\newtheorem{proposition}{命题}        % 创建一个新的定理环境,命名为 "proposition",并设置标题为 "命题"
\newtheorem{lemma}{引理}              % 创建一个新的定理环境,命名为 "lemma",并设置标题为 "引理"
\newtheorem{corollary}{推论}          % 创建一个新的定理环境,命名为 "corollary",并设置标题为 "推论"
\newtheorem{remark}{注解}             % 创建一个新的定理环境,命名为 "remark",并设置标题为 "注解"
\newtheorem{condition}{条件}          % 创建一个新的定理环境,命名为 "condition",并设置标题为 "条件"
\newtheorem{conclusion}{结论}         % 创建一个新的定理环境,命名为 "conclusion",并设置标题为 "结论"
\newtheorem{assumption}{假设}         % 创建一个新的定理环境,命名为 "assumption",并设置标题为 "假设"

% 文字边框
\usepackage[style=1]{mdframed}   % 带边框的文字块

% 列表格式控制
\usepackage{enumitem}            % 自定义有序列表格式(如 (1)、(2)、(3))

% 参考文献格式
\usepackage{gbt7714}             % 配置 GB/T 7714(中国国家标准)参考文献格式

% 附录管理(自动编号附录,并在目录中显示)
\usepackage[toc,page]{appendix}

% 图表相关
\usepackage{booktabs}                % 三线表(上粗中细下粗)
\usepackage{array}                   % 表格格式控制
\usepackage{tabularx}
\usepackage{diagbox}                 % 分类表头(对角线表头)
\usepackage{longtable}               % 长表格支持(允许表格跨页)
\usepackage{multirow}                % 加载 multirow 宏包,支持在表格中合并多行
\usepackage{float}                   % 控制图片浮动位置(H 选项可强制固定)
\usepackage{subfigure}               % 加载 subfigure 宏包,用于在同一行中插入多个子图
\usepackage{graphicx}                % 加载 graphicx 宏包,支持插入和缩放图像,常用于嵌入png、jpg等格式的图像
\usepackage{subcaption}
\usepackage{caption}                 % 加载 caption 宏包,用于自定义图表的标题和格式
\usepackage{color,xcolor}            % 加载 color 和 xcolor 宏包,支持文本、背景、框等内容的彩色设置
\captionsetup{labelsep=quad}         % 设置图形标题中标签和标题之间的分隔符为 "quad"(相当于一个空格宽度)
\renewcommand{\thesubfigure}{(\arabic{subfigure})}  % 重新定义子图编号的格式,使用圆括号括住阿拉伯数字编号
\renewcommand{\tabularxcolumn}[1]{m{#1}}
\newcolumntype{C}{>{\centering\arraybackslash}X}

% 脚注等符号相关
\usepackage{pifont}                   % 提供 Dingbat 符号,包括带圈数字(①②③…)
\usepackage[symbol*,stable]{footmisc} % 增强脚注功能,支持符号脚注
\DefineFNsymbols{circled}{            %自定义带圈数字作为脚注符号
  {\ding{192}}  % ①
  {\ding{193}}  % ②
  {\ding{194}}  % ③
  {\ding{195}}  % ④
  {\ding{196}}  % ⑤
  {\ding{197}}  % ⑥
  {\ding{198}}  % ⑦
  {\ding{199}}  % ⑧
  {\ding{200}}  % ⑨
  {\ding{201}}  % ⑩
}
\setfnsymbol{circled}                 % 让脚注默认使用带圈数字
\usepackage{wasysym}                  % 提供额外的符号(例如圆圈符号、勾号等)

% 设置公式、表格和图形的编号规则,使其包含章节号
\numberwithin{equation}{section}  % 使公式编号以章节号为前缀(例如:1.1,2.3)
\numberwithin{table}{section}     % 使表格编号以章节号为前缀(例如:1.1,2.3)
\numberwithin{figure}{section}    % 使图形编号以章节号为前缀(例如:1.1,2.3)

% 页边距调整(提高排版美观)
\usepackage{geometry}
\geometry{top=2.5cm,bottom=2cm,left=2.5cm,right=2cm}

% 页眉页脚设置
\usepackage{fancyhdr}          % 加载 fancyhdr 宏包,用于自定义页眉和页脚
\pagestyle{fancy}              % 启用 fancyhdr 宏包的自定义样式
\fancyhf{}                     % 清空页眉和页脚的默认内容
\fancyhead[C]{\zihao{5}  \songti 武汉理工大学毕业设计} % 设置页眉内容:宋体5号居中
\fancyfoot[C]{~\zihao{5} \thepage~}                           % 设置页脚内容:页码居中
\renewcommand{\headrulewidth}{0.65pt}                         % 设置页眉下方的横线宽度为0.65pt

\usepackage{titlesec}
\setlength{\baselineskip}{20pt}  % 设置段落的行距为 20 磅

% 设置标题的段前、段后间距
\titlespacing{\chapter}{0pt}{0.5\baselineskip}{0.5\baselineskip}  % 章标题
\titlespacing{\section}{0pt}{0.5\baselineskip}{0.5\baselineskip}  % 节标题
\titlespacing{\subsection}{0pt}{0.6\baselineskip}{0.5\baselineskip}  % 小节标题

% 章节标题格式设置
\ctexset{
    section = {                                         % 章节标题设置
        format = \centering\bfseries\zihao{-2} \heiti,  % 章节标题居中,使用黑体小二号
        name = {第, 章}                                 % 章节标题格式为“第X章”
    },
    subsection = {                                      % 小节标题设置
        format = \bfseries\zihao{3} \heiti          % 小节标题使用黑体三号
    },
    subsubsection = {                                   % 子小节标题设置
        format = \bfseries\zihao{4} \heiti          % 子小节标题使用黑体四号
    }
}

% 链接相关
\usepackage{hyperref}     % 加载 hyperref 宏包,支持文档中的交叉引用和超链接(如点击跳转)
\hypersetup{              % 设置链接颜色为黑色
	colorlinks=true,
	linkcolor=black,
	filecolor=black,      
	urlcolor=black,
	citecolor=black,
}

% 源代码和绘图相关
\usepackage{listings}                % 加载 listings 宏包,用于在 LaTeX 文档中粘贴源代码并高亮显示
\usepackage{tikz}                    % 加载 tikz 宏包,提供强大的图形绘制功能
\usepackage{tikz-3dplot}             % 加载 tikz-3dplot 宏包,扩展 tikz 用于绘制三维图形
\usetikzlibrary{shapes,arrows,positioning} % 加载 tikz 库,提供绘制形状、箭头、定位等功能
\usepackage[ruled]{algorithm2e}      % 伪代码,支持 if-else、for 等结构

% 重新定义文档元素的名称
\renewcommand{\contentsname}{目 ~~ 录}      % 将目录部分的标题修改为 "目  录" 
\renewcommand{\abstractname}{摘 ~~ 要}      % 将摘要部分的标题修改为 "摘  要"
\renewcommand{\refname}{参考文献}           % 将参考文献部分的标题修改为 "参考文献"
\renewcommand{\indexname}{索引}             % 将索引部分的标题修改为 "索引"
\renewcommand{\figurename}{图}              % 将图形部分的标题修改为 "图"
\renewcommand{\tablename}{表}               % 将表格部分的标题修改为 "表"
\renewcommand{\appendixname}{附录}          % 将附录部分的标题修改为 "附录"
\renewcommand{\proofname}{证明}             % 将证明部分的标题修改为 "证明"
\renewcommand{\algorithmcfname}{算法}       % 将算法部分的标题修改为 "算法"
%===================   结束导言   ===============================================================


%===================   正文区    ================================================================
\begin{document}

%============== 封皮和前言 =================
%===============  封面  =================
\smallskip
\begin{center}

\vspace*{2.2cm}
\zhongsong{\zihao{1} 武汉理工大学毕业设计} \\
\vspace*{3.3cm}
\heiti{\zihao{2} 基于U-Net的医学图像语义分割方法研究}\\
\vspace*{5.5cm}

\zhongsong
\begin{tabular}{cc}
 \zihao{-2} 学院(系):&\underline{\makebox[7cm][c]{\zihao{-2}自动化学院}} \\ 
 \\
 \zihao{-2}专业班级: & \underline{\makebox[7cm][c]{\zihao{-2}自动化2105}} \\ 
 \\
 \zihao{-2}学生姓名: & \underline{\makebox[7cm][c]{\zihao{-2}李子豪}} \\ 
 \\
 \zihao{-2}指导教师: & \underline{\makebox[7cm][c]{\zihao{-2}赵希}} \\ 
 \\
\end{tabular} 
\end{center}
\thispagestyle{empty}

\pagestyle{empty}      % 封面页不显示页码
\cleardoublepage       % 确保新页码从右页开始(书籍排版)
\pagenumbering{Roman}  % 切换为罗马数字
\setcounter{page}{1}   % 强制页码从Ⅰ开始
\pagestyle{plain} 
\section*{\zihao{-2} \centering 摘 ~~ 要}

\vskip0.5cm

本文基于注意力机制提出了改进的U-Net模型AAH U-Net,通过在网络中嵌入注意力门,同时采用数据增强和混合损失函数策略,AAH U-Net最终实现了在多个医学图像数据集上的语义分割性能提升。

针对U-Net网络中跳跃连接会将大量来自编码器的背景噪声传递至解码器的问题,本文提出了改进的跳跃连接设计,通过在跳跃连接中嵌入注意力门模块动态地筛选并加权特征图,加强对编码器特征图的特征提取,从而提高模型对目标区域的定位能力。

针对单一损失函数难以适应医学图像中普遍存在的类别不平衡和边界模糊问题,AAH U-Net采用了混合函数策略,通过使用Dice损失函数和交叉熵损失函数的加权混合损失函数,兼顾模型在前景区域重叠度与像素级分类准确率的表现,整体提升模型的分割性能和训练稳定性。

针对医学图像分割任务中常见目标结构复杂且样本规模有限的场景,AAH U-Net模型使用了数据增强策略,通过对图像引入翻转、随机旋转等几何变换,增强模型对空间变换的鲁棒性和小样本学习能力。

凭借注意力门机制、数据增强和混合损失函数策略,AAH U-Net实现了在皮肤癌镜像数据集、肝癌CT数据集和脑部肿瘤MRI数据集上语义分割性能的大幅提升。相较于原U-Net网络,AAH U-Net在皮肤癌镜像数据集上的验证集Dice系数从0.75提升至0.83,相对提升约11\%,同时在肝癌CT数据集和脑部肿瘤MRI上的验证集Dice系数分别为0.89和0.83,均大幅优于U-Net模型。

所有代码和训练好的AAH U-Net模型已开源至github:\url{https://github.com/WBlackhole/Black_essay}。

{\zihao{4} \heiti 关键词:} \zihao{-4}U-Net\quad 医学图像语义分割\quad 注意力机制\quad 卷积神经网络\quad AAH U-Net

\vskip0.5cm

\addcontentsline{toc}{section}{摘要}

\clearpage
\section*{\zihao{-2} \centering \textbf{Abstract} }

This paper proposes an improved U-Net model, AAH U-Net, based on attention mechanisms. By embedding attention gates into the network and employing strategies such as data augmentation and hybrid loss functions, AAH U-Net achieves enhanced semantic segmentation performance across multiple medical image datasets.

To address the issue that skip connections in the original U-Net architecture tend to pass a large amount of background noise from the encoder to the decoder, this paper introduces an improved skip connection design. Attention gate modules are embedded within the skip paths to dynamically filter and weight feature maps, thereby enhancing the extraction of meaningful features from the encoder and improving the model's ability to localize target regions.

To tackle the difficulty of single loss functions in handling class imbalance and blurry boundaries common in medical images, AAH U-Net adopts a hybrid loss strategy. By combining the Dice loss and cross-entropy loss in a weighted manner, the model simultaneously improves foreground overlap accuracy and pixel-wise classification performance, leading to better segmentation results and training stability.

Considering the complex anatomical structures and limited sample sizes often encountered in medical image segmentation tasks, the AAH U-Net model incorporates data augmentation techniques such as flipping and random rotation. This enhances the model’s robustness to spatial transformations and improves its few-shot learning capability.

With the combined benefits of attention gates, data augmentation, and hybrid loss strategies, AAH U-Net achieves significant improvements in semantic segmentation performance on skin cancer dermoscopy datasets, liver cancer CT datasets, and brain tumor MRI datasets. Compared to the original U-Net, the Dice coefficient on the skin cancer dataset increased from 0.75 to 0.83 (a relative improvement of approximately 11\%), while the validation Dice scores on the liver CT and brain MRI datasets reached 0.89 and 0.83, respectively—both substantially outperforming the baseline.

All source code and pretrained models of AAH U-Net are publicly available at GitHub: \url{https://github.com/WBlackhole/Black_essay}.

\textbf{\zihao{4} Key Words:} U-Net\quad Medical Image Segmentation\quad Attention Mechanism\quad Convolutional Neural Network\quad AAH U-Net

\addcontentsline{toc}{section}{Abstract}
\pagestyle{empty}
\tableofcontents 
\thispagestyle{empty}

%============== 论文正文   =================
\pagestyle{fancy}
\pagenumbering{arabic}
\section{绪论}

\subsection{研究背景}
%医学图像在疾病诊断、治疗规划和疗效评估中的重要作用。
%传统医学图像分割方法的局限性,引出深度学习在医学图像分割领域的兴起。

\subsection{研究意义}
%说明提高医学图像语义分割精度对于医学诊断和研究的积极意义。
%阐述本研究对现有U - Net方法改进的潜在价值。

\subsection{国内外研究现状}
%综述深度学习在医学图像分割领域的研究进展,重点介绍U - Net模型的提出、发展及其在不同医学图像分割任务中的应用情况。
%分析现有方法的优点和存在的问题。

\subsection{研究内容与创新点}
%明确本论文的研究内容,包括对U - Net网络结构的分析、改进方法的探索以及实验验证等。
%突出本研究的创新点,如对网络结构的优化、新数据增强方法的引入等。


\subsection{论文组织结构}
%简要介绍各章节的主要内容和逻辑关系。

\section{相关技术与理论基础}

\subsection{医学图像语义分割基础}

图像分割任务有两个不同的任务类别:语义分割任务和实例分割任务\cite{azad2024}。语义分割是像素级图像分割,图像中的每个像素点都有一个对应的类别,语义分割任务则需要尽量正确预测每个像素点的类别。实例分割在语义分割任务的基础上,需要将同类别的像素分类为不同的对象实例。基于医学图像的语义分割任务则是指利用计算机视觉技术去分析和处理2D或3D医学图像,达到将人体器官、软组织和病灶体进行分割、提取,三维重构和展示的目的\cite{liu2021}。

% 附图:语义分割和实例分割

\subsubsection{医学图像特点}

% 考虑是否引用第一章某处内容
医学图像作为临床诊断与治疗决策的重要依据,具备区别于自然图像的一系列独特特点:

\begin{enumerate}
    \item 多模态性: 
    多模态性是医学图像最本质的特征之一,不同成像技术基于各自的物理机制产生图像,导致图像在分辨率、对比度、噪声特性等方面具有显著差异。例如,CT图像通过X射线衰减反映组织的密度信息,表现为良好的骨骼成像与高密度区域识别能力;MRI通过核磁共振原理获取图像,具有出色的软组织对比度,常用于脑部、脊髓及肿瘤成像。
    
    \item 高噪声和低对比度:
    相较于多数自然图像的高清晰度和高对比度,高噪声与低对比度问题在医学图像中普遍存在。受限于成像设备、器官运动等因素,医学图像常伴有伪影与模糊,如MRI中常见的运动伪影与磁敏感伪影、CT中的金属伪影。低对比度则体现在图像的病灶区域与周围正常组织在灰度分布上高度重叠,如肝脏与肿瘤组织;也常出现多个器官重叠或粘连的情况,如腹部CT图像中的脾脏与左肾的接触区域、脑白质与灰质区域。
    
    \item 解剖结构的拓扑复杂:
    自然图像的物体通常具有清晰空间分层(如“前景-背景”),而在医学图像中器官和组织常呈现多层嵌套(如肠管缠绕、血管穿透脏器)和动态形变(如呼吸导致的肺位移)。同时,自然物体的形状相对稳定,而人体解剖结构会因为个体差异、病理状态产生巨大形态变异,例如前列腺癌患者的腺体体积可膨胀至正常值的3倍。
    
    \item 数据获取和标注困难:
    医学图像的数据获取和标注远高于自然图像。成像成本高、病人隐私问题、罕见病例少等等因素导致医学图像的数据量通常只有数百数千张,甚至更少。同时高质量的可用于图像分割的医学图像需要具有多年经验的临床专家进行逐像素级精细勾画,且不同医生间的标注标准与经验存在差异,导致“标注者间差异”现象普遍存在。

\end{enumerate}

% 附图:自然图像+高噪声医学图像+复杂拓扑结构医学图像

上诉特点使得医学图像相较于自然图像有更高的复杂性,对其的自动分割任务不仅要求模型具备强大的特征表达与结构建模能力,还需应对多种极具挑战性的实际问题。

\subsubsection{医学图像语义分割挑战}

在数据具有多模态性和获取标注困难的情况下,跨模态泛化能力成为了衡量模型在医学图像语义分割性能的关键评估指标,一个鲁棒的能够适用于医学图像语义分割任务的模型应当能够在这种模态一致性不足和低数据量的情况下维持良好稳定的性能。而低对比度和解剖结构复杂的特点,对模型提出了更强的捕捉高分辨率细节并维持边缘清晰度的能力要求,许多早期病灶(如肺部结节、肝癌早期病变)和器官(如视网膜微血管)在图像中的所占面积比例极小,容易在下采样的过程中被忽略和抹除,模型需要能够在这种情况下对小目标进行精细分割。

除了受医学图像固有的复杂性影响,高实时性也是医学图像语义分割模型应当具备的能力,同时也是决定模型能否落地应用的重要门槛。在术中导航、放疗定位、超声实时识别等高时效性场景中,模型不仅要保证分割准确,还需在极短时间内完成推理,这同时也对模型的计算复杂度与硬件适配能力提出了极高的要求。

综上,面向于医学图像的语义分割算法和模型不仅有高分割精度和跨模态泛化的能力要求,还需要面对高实时性等的性能挑战。这种挑战也促使了大量的研究者进入医学图像语义分割领域进行探索,提出了诸多有效的分割算法和创新的网络结构与优化策略,这些工作将在下文中进行全面的介绍。

\subsection{传统分割方法概述}

在相当长的一段时间内,传统分割技术作为医学图像分析的基础提供了一系列复杂性和适用性有所不同的方法,包括阈值分割、边缘检测、区域生长及聚类分割等在内的传统分割方法在很多场合下已被证明有效\cite{xu2024}。接下来的内容将对阈值分割、边缘检测和聚类分割这三种最具代表性且广泛应用的传统分割方法进行概述总结。

\subsubsection{阈值分割}

阈值分割是图像分割中最基础、最广泛使用的技术之一,其核心思想是通过设定一个或多个灰度阈值,将图像像素进行二分类(如前景和背景):

\begin{equation}
B(x, y)=\left\{\begin{array}{ll}1, & \text { if } I(x, y) \geq T \\ 0, & \text { if } I(x, y)<T\end{array}\right.
\end{equation}

其中,$I(x, y)$表示图像在坐标$(x, y)$处的像素值,$B(x, y)$表示二元图像在坐标$(x, y)$处的像素值,$T$则是阈值。在图像处理中,阈值分割方法又可以细分为两大类:全局阈值法和局部阈值法。

全局阈值法假设整幅图像的背景与前景有明显的灰度差异,因此通过应用一个统一的阈值$T$来对图像进行分割。比较著名的全局阈值算法包括:Otsu法、迭代法和最小误差法。Otsu法通过最大化类间差异选取最优阈值;迭代法则初始估计一个阈值,然后迭代更新前景和背景的平均灰度值来计算新阈值直至收敛;最小误差法在基于图像前景和背景的像素灰度值呈正态分布的假设前提下,通过计算各分割区域的概率密度函数得到总分类误差并使误差最小化确定阈值。全局阈值法在图像灰度分布较均匀、对比度明显时表现良好,且计算成本低易于实现和集成到实时系统。但是,该方法在光照不均或对比度不明显的情况下分割效果差,同时也不适用于复杂背景或多目标图像分割。

相对于全局阈值法,局部阈值法假设图像具有不均匀的光照,因此针对每个像素的局部领域单独计算阈值$T$。较为常见的局部阈值算法包括:Niblack算法和Sauvola。Niblack通过计算每个像素上特定窗口内像素值的局部平均值和标准偏差,并利用平均值评估局部亮度、标准偏差衡量对比度和纹理来动态设定阈值:$ T=\mu+k \sigma $。Sauvola则在Niblack的基础上进行改进,将标准偏差的动态范围R引入阈值计算:$ T=\mu\left[1+k\left(\frac{\sigma}{R}-1\right)\right] $,从而更好地处理变化的背景和光照条件。局部处理使得局部阈值法能处理光照不均、阴影遮挡的情况,提高了图像细节的保留能力,使其能适用于安全监控、指纹识别等对细节要求较高的场景。但是同时,其计算复杂度也变得更高、运行时间较长,而且在噪声较大的区域容易误判。

\subsubsection{边缘检测}

% 记得插入参考文献

边缘是是指两个均匀区域之间的边界,表现为图像中像素强度的局部突变。而边缘检测的核心就在于识别定位图像中像素强度突变的位置。在基于边缘的分割方法中,首先检测图像中目标物体的轮廓以及物体与背景之间的边界,随后通过连接边缘形成物体边界以实现目标区域分割。

在基于边缘的图像分割技术中,整个过程通常按一系列明确的步骤展开。首先进行图像预处理,为后续分析奠定基础;随后执行边缘检测这一关键步骤,识别图像中的潜在边界;接着应用非极大值抑制技术对检测结果进行精细化处理,强化显著边缘的同时抑制微弱响应;然后通过阈值处理将梯度图二值化,实现边缘与非边缘的明确区分;之后是后处理阶段,通过填补间断或平滑不规则边缘进一步优化分割结果;最终依据上述步骤所得的边缘信息完成图像的区域分割。

图像边缘具有两个基本特性:方向与幅值。沿着边缘方向,像素值的变化往往较为平缓;而在垂直于边缘方向,像素值的变化则较为剧烈。基于这一特性,边缘检测常应用一阶和二阶微分算子来描述和检测图像中的边缘。

常见的一阶微分算子包括Roberts交叉算子、Prewitt算子、Sobel算子以及Canny边缘检测器,这些算子通过突出图像中像素强度的梯度变化来实现边缘检测。而二阶微分算子如Laplacian算子和高斯拉普拉斯算子同样被广泛采用。这些算子通过将特定模板与图像像素值矩阵进行卷积运算,精确计算各像素点的梯度变化或曲率,从而提取边缘特征。

如表xx所示,不同的边缘检测算子具有各自的优势与局限性,适用于不同的应用场景。例如,Roberts交叉算子因其核尺寸小而具有计算简单、速度快的优势,特别适用于低噪声图像,但其对噪声敏感且在边缘定位精确方面表现仅达中等水平。

\begin{table}[htbp]
    \centering
    \caption{边缘检测算子性能对比}
    \label{tab:edge_operators}
    \begin{tabular}{>{\raggedright}p{3cm}>{\raggedright}p{4.5cm}>{\raggedright}p{4.5cm}}
        \toprule
        \textbf{算子类型} & \textbf{优势} & \textbf{劣势} \\
        \midrule
        Roberts Cross & 计算简单,在低噪声场景下效果显著 & 对噪声敏感,边缘定位精度不足 \\
        \addlinespace[0.2cm]
        Prewitt & 对边缘方向特别敏感 & 抗噪性差,边缘清晰度较低 \\
        \addlinespace[0.2cm]
        Sobel & 与Prewitt类似但具有更好的噪声抑制能力,同时提供适度的边缘平滑 & 可能导致边缘过度模糊,不适用于高精度边缘检测任务 \\
        \addlinespace[0.2cm]
        Canny & 边缘检测精度高,能有效抑制噪声并精确定位边缘 & 计算复杂度高,性能高度依赖参数选择 \\
        \addlinespace[0.2cm]
        Laplacian & 能准确定位边缘中心点,对细节变化敏感 & 极易受噪声干扰,且无法提供边缘方向信息 \\
        \addlinespace[0.2cm]
        Laplacian of Gaussian & 高斯预平滑有效降低噪声影响,边缘定位性能优良 & 计算复杂度较高,可能丢失部分细微边缘 \\
        \bottomrule
    \end{tabular}
\end{table}


Prewitt算子与Sobel算子均采用像素邻域差分法进行边缘检测。其中,Prewitt算子对边缘方向具有更高的敏感性,而Sobel算子则通过改进的噪声抑制能力获得更优性能,但代价是可能导致边缘轻微模糊。Canny算子作为多阶段边缘检测算法的典型代表,以高精度著称,不仅能有效抑制噪声干扰,还可实现边缘的准确定位,但其计算复杂度显著高于其他算子。

基于二阶导数的Laplacian算子对图像细节具有突出增强作用。尽管该算子易受噪声影响,但在细微特征增强方面表现卓越。高斯拉普拉斯算子通过高斯平滑预处理与Laplacian算子的协同作用,有效降低了噪声干扰,特别适用于噪声敏感环境下需要精确定位边缘的应用场景。

\subsubsection{聚类分割}

\subsection{基于深度学习的语义分割技术}

\subsubsection{卷积神经网络}

\subsubsection{U-Net网络}


\subsection{模型评估指标}
\section{基准U-Net模型设计与实现}

\subsection{数据集准备与预处理}

\subsubsection{ISBI皮肤镜影像数据集}

ISBI皮肤镜影像数据集(2018年)由ISIC联合Memorial Sloan Kettering Cancer Center、University of Queensland等多家机构共同构建,并作为ISIC 2018皮肤病变识别挑战赛的官方数据资源\cite{codella2019skinlesionanalysismelanoma}。

ISIC 2018数据集作为目前最具影响力的皮肤镜数据集之一,广泛用于皮肤病变分割模型的基准测试。数据集总共包含2594张RGB三通道皮肤镜图像,图像格式为JPEG或PNG,分辨率在600×450至6748×4499像素之间,保留了丰富的病变细节特征。在标注方面,数据集提供由皮肤科专家精确勾画的像素级病变掩膜用于病变区域分割任务训练,掩膜的标注一致性通过交叉验证确认(Kappa 系数大于 0.82)。同时,整个数据集按照任务需求被官方划分为了训练集(2,000 张,含掩膜和标签)、验证集(300 张,仅标签)与测试集(294 张,无公开标注,仅用于模型评估)。

为确保输入数据的一致性与模型训练的稳定性,本研究对ISIC 2018皮肤镜影像数据集中的图像及其对应的掩膜进行了标准化预处理。通过随机分层抽样划分数据集(训练集:验证集:测试集=7:2:1),同时所有图像的像素值统一归一化至$(0,1)$范围,掩膜则转换为双通道概率标签(0-背景,1-病变),为后续使用基于概率分布的损失函数(如交叉熵损失、Dice损失)提供结构支持。

\subsubsection{LiTS肝脏CT数据集}

LiTS(Liver Tumor Segmentation)数据集是由MICCAI 2017肝脏肿瘤分割挑战赛(Liver Tumor Segmentation Challenge 2017)发布的医学影像公开数据集,由来自全球多家权威医疗机构提供的腹部增强CT扫描组成,涵盖130例临床病例,包含多种肝脏病理状态,包括肝细胞癌、转移性肿瘤、胆管细胞癌等\cite{Bilic_2023}。所有图像数据均采用静脉期CT成像,层厚范围为1–5mm,横断面矩阵分辨率为512×512,体素间距在0.6–1.0 mm之间。

在标注方面,数据集中每一病例均由三位经验丰富的放射科专家进行独立标注,提供像素级肝脏与肿瘤分割掩膜。标注一致性通过Dice系数(平均大于0.92)和Hausdorff距离(平均小于5 mm)进行验证,部分病例还附带有病灶的病理学分型标签。

LiTS 2017数据集存在一些区别于ISBI 2018数据集的挑战:其一,肝脏轮廓复杂、变异性大,且边界常与胃肠道等邻近脏器重叠,导致区域判别困难;其二,肿瘤病灶形态异质性强(如结节型、融合型、浸润型),并存在大小悬殊、边界模糊、低密度病灶等不利因素;此外,呼吸运动伪影与层间不连续性也显著增加了分割任务的难度。

在模型训练前,本研究对LITS 2017肝脏CT数据集进行了系统性预处理,以确保数据一致性与模型输入的可靠性。所有CT图像以伪彩色映射形式统一读取为RGB三通道格式,并转换为浮点张量,数值范围归一化至$(0,1)$;对应的单通道掩膜(0表示背景,1表示肝脏及肿瘤区域)同步转换为张量格式,针对二分类任务需求,掩膜进一步编码为双通道概率标签(通道0为背景概率,通道1为前景概率),以适配基于概率分布的损失函数。同时,数据集严格遵循预定义划分策略,通过固定随机种子生成可复现的训练集、验证集与测试集分配方案。整个的数据集处理流程通过标准化映射、严格划分与一致性校验,为肝脏肿瘤分割任务构建了高鲁棒性的数据基础。

\subsubsection{BraTS脑肿瘤MRI数据集}

BraTS(Brain Tumor Segmentation)2020 数据集由 MICCAI 组织联合宾夕法尼亚大学、慕尼黑工业大学等多家国际顶尖医学与工程研究机构共同构建,作为年度脑肿瘤分割挑战赛的重要基准数据资源\cite{menze2015}。该数据集专注于胶质瘤(包括高级别胶质瘤HGG与低级别胶质瘤LGG)MRI图像的分割任务,广泛用于评估多模态影像分割算法的性能与鲁棒性。

数据集中共包含369例患者的多模态MRI扫描数据,每例数据均包含四种标准MRI模态:T1加权、T1对比增强(T1ce)、T2加权以及FLAIR。在数据标注方面,每例数据均提供由神经放射学专家联合标注的像素级三维分割掩膜,标注内容涵盖:

\begin{enumerate}
    \item 增强肿瘤区域(Enhancing Tumor,ET):表现为对比增强T1序列中具有强化表现的病灶区域;
    \item 肿瘤核心(Tumor Core,TC):包括坏死区域、实性肿瘤与增强部分;
    \item 肿瘤整体(Whole Tumor,WT):包括肿瘤核心与周围水肿区域。
\end{enumerate}

BraTS 2020数据集的主要挑战包括:(1)肿瘤组织的高度异质性,如增强区与坏死区边界模糊;(2)小体积或卫星灶的检测难度高,极易被误判或遗漏;(3)多模态之间的病灶表征差异显著,对模型特征融合能力提出更高要求。

为构建高效且可复现的脑肿瘤分割数据管线,本研究基于BraTS 2020数据集设计了系统化预处理流程。首先,所有病例按目录名排序并剔除损坏样本,采用两级随机划分策略:10\%病例作为独立测试集,剩余病例在固定随机种子控制下进一步分为训练集(70\%)与验证集(20\%),病例ID分配方案持久化为JSON文件以确保实验可复现性并避免数据泄漏。

有研究表明对于脑肿瘤语义分割任务,FLAIR和T1ce这两种模态组合作为输入序列是最佳的组合选择\cite{buchner2023}。因此,本研究仅采用FLAIR和T1ce两种模态组合作为输入序列。此外,所有输入数据逐例归一化至$(0,1)$范围以消除扫描设备差异,并遍历三维体积数据的轴向切片,剔除无标注信息的空白切片以缓解类别极端不平衡问题;对于原始标签则转换为多通道one-hot编码,适配交叉熵损失与Dice损失的监督需求。

\subsection{基准U-Net模型构建}

本研究在经典U-Net框架的基础上实现了基准U-Net模型的构建,用以评估后续改进策略的真实收益。模型构建包括网络层的搭建、损失函数的涉及和优化器的选择。

\subsubsection{网络层}

图~\ref{fig:unet_ushape}展示了本研究搭建的基准U-Net模型架构,采用对称的编码器—解码器结构:

编码端连续堆叠四级特征抽取单元,每一级由两层3×3有填充卷积与ReLU激活组成;特征通道数以32为起始,在每次下采样后按2倍的倍率递增(即32→64→128→256)。空间下采样通过2×2最大池化实现,使特征图尺寸依次减半,从而在更大的感受野上编码语义信息。

解码端采用与编码端对称的结构,对最深层特征进行转置卷积上采样,并与对应级别的编码器输出通过跳跃连接进行特征级串接后,经过两个卷积层进行卷积以还原局部细节。最终输出层采用1×1卷积,将通道数映射为像素的语义类别数后得到输出。

\begin{figure}[h]
    \centering
    \includegraphics[width=0.6\textwidth]{fig/Unet_ushape.png}
    \caption{基准U-Net模型架构}
    \label{fig:unet_ushape}
\end{figure}

上述网络拓扑及通道配置均采用Pytorch框架实现,对应的实现要点可在附录A中进一步查证。

\subsubsection{损失函数的设计}

在损失函数的设计上,本文采用Dice和Cross-Entropy的混合损失函数,即以相等权重线性叠加Dice损失与像素级交叉熵(CE)损失,以充分兼顾类别不平衡下的区域重叠度优化与梯度稳定性。设网络输出的类别概率图(像素总数为$N$)为:$ P=\left\{p_{i}\right\}_{i=1}^{N} $,真实分割掩膜为$ G=\left\{g_{i}\right\}_{i=1}^{N} $,其中$ p_{i} \in[0,1], g_{i} \in\{0,1\}$分别是像素$i$的前景概率和真实标签。则:

\begin{equation}
    \mathcal{L}_{\text {Dice }}=1-\frac{2 \sum_{i=1}^{N} p_{i} g_{i}}{\sum_{i=1}^{N} p_{i}+\sum_{i=1}^{N} g_{i}+\varepsilon}
\end{equation}

\begin{equation}
    \mathcal{L}_{\mathrm{CE}}=-\frac{1}{N} \sum_{i=1}^{N}\left[g_{i} \ln \left(p_{i}\right)+\left(1-g_{i}\right) \ln \left(1-p_{i}\right)\right]
\end{equation}

\begin{equation}
    \mathcal{L}_{\text {mix }}=0.5 \mathcal{L}_{\text {Dice }}+0.5 \mathcal{L}_{\mathrm{CE}}
\end{equation}

其中$ \varepsilon=10^{-6} $用于数值平滑以防零分母。Dice项直接对预测与标注的重叠区域进行归一化度量,能在小体积病灶场景下显著提升召回;交叉熵项则提供像素级对数似然的密集监督,改善早期训练阶段梯度稀疏、收敛震荡等问题。

\subsubsection{优化器的选择}

优化器采用Adam随机梯度下降算法,其一阶自适应动量可在早期快速探索有效学习率区间,同时在震荡平衡阶段保持较小更新幅度。初始学习率设定为$ 1.0 \times 10^{-4} $,一阶、二阶动量系数沿用默认值$ \beta_{1}=0.9, \beta_{2}=0.999 $。

\subsection{训练策略与评估指标}

\subsubsection{训练策略}

模型的训练均在Kaggle平台进行,采用Kaggle平台提供的NVIDIA P100 (16 GB) 单卡进行模型加速训练。

为了避免训练后期进入平台期,附加 Reduce-on-Plateau 学习率调度策略:当验证集指标在 10 个 epoch 内无显著提升时,将学习率衰减为原值的 0.5;最低学习率限定在$ 1.0 \times 10^{-6} $,防止数值下溢而不再更新参数。当学习率因多次衰减而触碰下限时,若模型仍无改善,则提前终止训练以节省算力。

训练超参数经预实验网格搜索确定:批量大小根据数据集的规模和大小确定,完整训练周期设置为100epoch,保证验证过程与训练同步,本研究在每个 epoch 结束后立即于验证集评估 Dice 系数并记录最优权重;测试集仅使用单尺度推断,不采用多尺度或模型集成,确保基准模型公平、简洁、可复现。

\subsubsection{评估指标}

在医学图像分割研究中,模型输出通常为与输入图像同尺寸的二值概率图,衡量其与专家标注掩膜之间的相似程度是评价算法优劣的关键。单一指标无法满足同时反映检测率、误诊率和重叠精度的要求,多指标可立体呈现模型表现。因此,在本研究中,我们以混淆矩阵四要素(真阳性TP、真阴性TN、假阳性FP、假阴性FN)为基础,记录Accuracy、Precision、Recall、Specificity、F1-Score、Dice系数与Jaccard指数七项评估指标。下面逐一给出这些指标的定义、公式及其评价意义。

准确率Accuracy表示正确预测的样本占总样本的比例,是最直观的分类性能指标:

\begin{equation}
    \mathrm{Accuracy}=\frac{TP+TN}{TP+TN+FP+FN}
\end{equation}

Accuracy直观反映模型的综合判断能力,但在病灶面积远小于背景时,Accuracy易受TN主导,可能高估模型质量,因此仅作为参考基线。

特异性Specificity表示实际为负类的样本中被正确预测的比例,衡量模型识别阴性样本的能力:

\begin{equation}
    \mathrm{Specificity}=\frac{T N}{T N+F P}
\end{equation}

在医学任务中,高特异性可减少健康组织被误判为病变的风险(如避免正常脑组织被误分割为肿瘤),提升结果的可信度,与 Recall 形成互补。

Jaccard指数(IoU)计算预测与真实标签的交集与并集的比值,是分割任务的经典指标:

精确率Precision表示预测为正类的样本中实际为正类的比例,反映模型的预测可靠性:

\begin{equation}
    \mathrm{Precision}=\frac{T P}{T P+F P}
\end{equation}

Precision反映模型整体预测正确性,但在类别极度不平衡时(如背景像素占比90\%以上),可能高估性能(例如模型仅预测背景即可获得高准确率),需结合其他指标综合判断。

召回率Recall召回率表示实际为正类的样本中被正确预测的比例,反映模型对正类样本的覆盖能力:

\begin{equation}
    \mathrm{Recall}=\frac{T P}{T P+F N} 
\end{equation}

高召回率意味着模型能有效捕捉病变区域(减少漏检),在早期诊断(如癌症筛查)中至关重要,但需平衡精确率以避免过度预测。

F1-Score是Precision与Recall的调和平均数,综合衡量模型在正类样本上的分类能力,尤其适用于类别不平衡场景:

\begin{equation}
    \mathrm{F} 1=\frac{2 T P}{2 T P+F P+F N}=2 \cdot \frac{\text { Precision } \times \text { Recall }}{\text { Precision }+ \text { Recall }}
\end{equation}

F1分数可以避免仅关注单一指标(如高精确率但低召回率),在医学图像分割中(如肿瘤区域占比小),能更均衡评估模型对正类(病变区域)的捕捉能力与预测准确性。

Dice系数衡量预测结果与真实标签的重叠程度,是医学图像分割的核心评估指标:

\begin{equation}
    \mathrm{Dice}=\frac{2 T P}{2 T P+F P+F N}=\frac{2 \times|A \cap B|}{|A|+|B|}
\end{equation}

其中,A为预测区域,B为真实区域。Dice系数直接衡量空间重叠,对不平衡数据敏感度低,能有效评估模型对病灶轮廓的捕捉精度,尤其适用于小目标分割任务(如脑肿瘤核心)。

\begin{equation}
    \mathrm{Jaccard}=\frac{T P}{T P+F P+F N}=\frac{|A \cap B|}{|A \cup B|}
\end{equation}

其中,A为预测区域,B为真实区域。Jaccard指数严格量化重叠区域的比例,对分割边界的轻微偏移敏感,常用于评估模型在复杂解剖结构(如肿瘤浸润区域)中的细节保留能力,在多模型比较时可提供更保守的评估视角。
\section{改进的U-Net模型设计}

% 总领起始段

\subsection{网络的改进设计}

\subsubsection{改进设计的动机}

尽管U-Net网络的跳跃连接能够有效地结合浅层和深层特征,但直接将编码器浅层特征映射与对应解码器特征级联,这种“无差别”拼接会把大量与前景无关的背景噪声一并传递,造成解码器在处理边界模糊和极小目标时,可能因为上下文信息的缺失而导致定位不准确。为了抑制冗余背景,同时保持高分辨率的边缘信息,本研究提出了改进的U-Net模型设计,在每一级跳跃通路中引入注意力门,依据编码器高层语义对编码器特征进行动态像素级筛选。

注意力机制的核心思想是通过动态地调整特征图的权重,自动聚焦于目标区域,抑制无关背景的干扰。这种方法能够增强模型对重要区域的敏感度,提高模型对小目标的定位能力\cite{oktay2018}。具体而言,注意力门根据输入特征图和从粗尺度上提取的上下文信息,计算每个像素的注意力系数,并利用该系数对特征图进行加权,从而只保留对分割任务有用的区域。这一策略无需额外的外部局部化模型,通过自动学习重要区域,有效提升了模型的分割精度,同时避免了传统多阶段模型中冗余计算和参数过多的问题。

%在网络结构中的嵌入位置上,注意力模块被集成到U-Net的跳跃连接部分。传统的U-Net通过将编码器的低层特征图与解码器高层特征图进行拼接来进行信息传递,而在引入注意力机制后,注意力模块被放置在编码器和解码器跳跃连接的中间。在这一位置,注意力模块可以根据解码器的上下文信息来加权编码器的特征图,确保网络关注到最重要的区域并忽略无关背景。具体而言,在每个跳跃连接处,编码器的特征图 $x_l$ 和解码器的特征图 $g_l$ 会被送入注意力门模块,经过加权后再与解码器的上采样特征进行拼接。这一设计使得注意力机制能够在局部区域进行动态学习,从而提高了分割的精度,尤其是在分割小目标和复杂背景时的表现。

\subsubsection{注意力门模块的结构与原理}

%此外,注意力模块采用的网格注意力机制(Grid Attention Mechanism),相较于传统通道注意力(如SENet)仅关注通道维度,本研究的网格注意力机制通过局部区域的动态调整,兼顾空间与通道信息,更适用于医学图像中目标形态多变的场景。

注意力门模块的内部结构主要由三个关键部分组成:输入特征的加权过程、注意力系数的计算以及门控机制的输出。具体而言,该模块包括权重矩阵 $W_x$ 和 $W_g$、非线性激活函数 ReLU、Sigmoid 激活函数以及注意力权重的计算等组成部分。

首先,注意力门模块接收来自U-Net编码器的特征图 $x_l$ 和解码器的门控信号 $g_l$ 作为输入。$x_l$ 为编码器第 $l$ 层输出的特征图,$g_l$ 则是来自解码器的特征图,它为后续的注意力加权提供上下文信息。为了将这两者的特征进行融合,模块采用了两种1×1卷积操作:一个用于处理编码器的特征图 $x_l$,另一个用于处理解码器的门控信号 $g_l$。这两者都通过对应的卷积核 $W_x$ 和 $W_g$ 进行映射到一个中间空间,从而保持信息的一致性并准备后续的注意力计算。

接下来,在这两路特征图被处理后,它们通过ReLU激活函数进行非线性转换,这一步骤有助于引入非线性特征表达,使得网络能够更好地学习到复杂的关系。此时,经过ReLU激活的特征图被送入一个加法操作,进行特征融合,得到一个结合了编码器和解码器信息的中间表示。为了进一步处理该表示并生成最终的注意力权重,模块通过Sigmoid激活函数计算得到一个归一化的注意力系数 $\alpha_l$,该系数决定了每个像素的权重。最后,这些权重通过逐元素相乘的方式与编码器特征图 $x_l$ 进行加权,从而生成加权后的特征图 $\tilde{x}_l = \alpha_l \cdot x_l$。

在计算过程中,注意力系数 $\alpha_l$ 的计算公式如下:

\begin{equation}
    q_{\text{att}}^l = \psi^T \left( \sigma_1 (W_x^T x_l + W_g^T g_l + b_g) \right) + b_\psi
\end{equation}

\begin{equation}
    \alpha_l = \sigma_2 \left( q_{\text{att}}^l(x_l, g_l; \Theta_{\text{att}}) \right)
\end{equation}

其中,$\sigma_1$ 和 $\sigma_2$ 分别为 ReLU 和 Sigmoid 激活函数,$W_x$ 和 $W_g$ 是用于特征映射的权重矩阵,$\psi$ 是线性变换矩阵,$b_g$ 和 $b_\psi$ 是偏置项。在这个公式中,$q_{\text{att}}^l$ 是通过加法操作和线性变换后得到的注意力得分,$\alpha_l$ 是最终计算得到的注意力系数。通过Sigmoid激活,$\alpha_l$ 的值范围被限制在$[0, 1]$之间,能够根据不同位置的特征重要性动态调整每个位置的注意力权重。

为了进一步探究注意力门的机制和作用,本文直观展示了同一张 ISIC 图像分别在原始 U-Net 与 Attention-U-Net 四级跳跃通路上的特征图,并截取其中 4 个通道进行并排可视化,如图 \ref{fig:skip_vs_gate_vis} 所示。图中左半四列(Skip:ch0-ch3)对应原始U-Net网络的编码器跳跃连接后准备和解码器拼接的特征图;右半四列(Gate:ch0-ch3)则是 Attention U-Net经过注意力门调制后的准备和解码器拼接的特征图。纵向从上到下依次为四级跳跃连接(Skip1/Gate1-Skip4/Gate4)。

\begin{figure}[!htbp]
    \centering
    \includegraphics[width=\textwidth]{fig/unet_vs_attunet_feature_compare.png}
    \caption{}
    \label{fig:skip_vs_gate_vis}
\end{figure}

通过对比,可以得到三点关键观察:

\begin{enumerate}
    \item 浅层抑噪:在 Skip1 与 Skip2 中,原始特征几乎整幅图像均有弱激活,背景纹理与病灶边界难以区分;而 Gate1、Gate2 中大量背景像素被压制至 0 – 0.2 的低激活区,仅在病灶主体及其边缘保留显著响应,说明注意力门已在最早层面过滤掉无用纹理。
    
    \item 深层聚焦:随着网络加深(Skip3、Skip4),原始跳跃特征仍夹杂噪声条纹,而 Gate3、Gate4 的高亮区域收缩并对齐于病灶外轮廓,形成清晰的连贯带;由此可见,门控权重不仅抑制背景,还在深层强化对判别性边界的响应。
    
    \item 信息重加权而非简单剪裁:Gate-ch 通道在病灶区域的激活强度显著高于 Skip-ch(色条 0.6 – 1.0),而非单纯“置零”背景后保持原强度。这表明注意力门起到了增益—抑制双向调制 的作用:一方面降低噪声,另一方面放大关键特征,从而在解码阶段提供更清晰的边界信息。
\end{enumerate}

上述可视化与前文对注意力门工作机制的理论分析一致,也为后续消融实验(见第 4.2 节)将呈现的定量改进提供直观佐证:在加入注意力门后,我们将看到验证集 Recall 和 Dice 均显著提高,而 Precision 仅出现轻微波动。这说明注意力门能够在跳跃连接中对高分辨率特征进行“像素级甄别”,在最大限度保留判别性信息的同时,有效压制背景噪声,从源头减少漏检而不过度引入误检。

\subsection{损失函数的改进设计}

在损失函数的设计上,改进的U-Net模型采用Dice Loss和Cross-Entropy Loss的混合损失函数,即以相等权重线性叠加Dice损失与像素级交叉熵(CE)损失,以充分兼顾类别不平衡下的区域重叠度优化与梯度稳定性。设网络输出的类别概率图(像素总数为$N$)为:$ P=\left\{p_{i}\right\}_{i=1}^{N} $,真实分割掩膜为$ G=\left\{g_{i}\right\}_{i=1}^{N} $,其中$ p_{i} \in[0,1], g_{i} \in\{0,1\}$分别是像素$i$的前景概率和真实标签。则:

\begin{equation}
    \mathcal{L}_{\text {Dice }}=1-\frac{2 \sum_{i=1}^{N} p_{i} g_{i}}{\sum_{i=1}^{N} p_{i}+\sum_{i=1}^{N} g_{i}+\varepsilon}
\end{equation}

\begin{equation}
    \mathcal{L}_{\mathrm{CE}}=-\frac{1}{N} \sum_{i=1}^{N}\left[g_{i} \ln \left(p_{i}\right)+\left(1-g_{i}\right) \ln \left(1-p_{i}\right)\right]
\end{equation}

\begin{equation}
    \mathcal{L}_{\text {mix }}=0.5 \mathcal{L}_{\text {Dice }}+0.5 \mathcal{L}_{\mathrm{CE}}
\end{equation}

其中$ \varepsilon=10^{-6} $用于数值平滑以防零分母。Dice项直接对预测与标注的重叠区域进行归一化度量,能在小体积病灶场景下显著提升召回;交叉熵项则提供像素级对数似然的密集监督,改善早期训练阶段梯度稀疏、收敛震荡等问题。

\subsection{训练策略的改进设计}

根据Marire-Hein等人\cite{maier-hein2018a}的统计,大量医学图像挑战赛的训练集大小平均集中在100~400之间。对于本研究使用的三个数据集:ISIC、LiTS和BraTS,其情况和特点已在3.1节进行详细阐述。从数据集规模看,三个数据集在样本数量上均已达到医学图像语义分割的中等规模。然而对于医学图像语义分割任务,在医学图像的目标结构和背景环境都较为一致的情况下,模型的训练容易学习到固定的像素分布模式,而非像素语义概念,进而使得模型对输入数据的轻微旋转、形变和灰度偏移变得十分敏感。

基于上诉分析,单纯的依赖原始样本不足以支撑有效的特征学习,在这种情况下,采用数据增强策略是提升模型性能和稳定性的关键。在对多项医学图像分割竞赛模型的系统分析中,Maier-Hein等人\cite{maier-hein2018a}指出,几乎所有方法都采用了数据增强策略,尤其是在病例数量在200+左右时。此外,Isensee等人在提出nnU-Net网络时也明确强调\cite{isensee2021},即使在中等数据规模的数据集上,不做数据数据增强也会严重影响模型验证集的性能。

因此,本研究在改进的U-Net模型设计中引入了包括随机旋转、水平垂直翻转以及弹性形变的数据增强训练策略,以帮助模型提升对于图像取向、灰度变化和形态扰动的鲁棒性和学习有泛化能力的语义特征表示。
\section{实验与结果分析}

\subsubsection{实验设置与对比方法}

本文所有实验均在基于Ubuntu 20.04操作系统的Kaggle Notebook环境完成,硬件配置采用Kaggle提供的单张NVIDIA P100 GPU(16GB显存)和30GB RAM内存。实验模型代码基于PyTorch框架搭建,同时搭配CUDA 11.3加速库加速模型训练。

模型的网络权重的初始化采用He normal,并使用初始学习率为$1 \times 10^{-4}$的Adam优化器更新权重。每个实验都进行100轮(epoch)实验,且在每轮训练结束后在验证集上进行模型评估,保存Dice得分最高的模型权重。

本文在进行模型评估时,以Dice系数作为核心评价指标,用于衡量模模型在分割任务中预测结果与真实标签之间的重叠程度,是医学图像分割中最常用且最敏感的评估指标之一。为了更全面地反映模型性能,辅以Jaccard指数、F1分数、准确率(Accuracy)、精确率(Precision)与召回率(Recall)等多维度指标进行综合评估。

\subsection{消融实验}

为了系统评估各改进模块对模型语义分割性能的影响,本研究设计了一系列消融实验,围绕模型结构、损失函数与训练策略三个层面展开。以基准U-Net模型作为对照组,逐步引入或移除关键组件,包括跳跃连接、注意力机制、数据增强策略、以及不同的损失函数组合,来观察每项设计对模型性能的独立贡献与协同增益。

所有消融实验均基于ISIC 2018皮肤癌图像分割数据集进行,采用 8:2 比例划分训练集与验证集,并在固定的模型训练框架和超参数设置下进行公平比较。通过精心设计的分组实验与逐项对照分析,本节将展示各模块在模型收敛速度、最终性能、错误类型等方面的具体影响,为构建最终优化方案提供理论依据与实证支撑。

\subsubsection{基准模型性能验证}

在消融实验中,基准U-Net模型使用原始U-Net网络结构,配合混合损失函数和Adam优化器进行训练,关于损失函数和优化器的具体参数设置已在第三章阐述。此外,基准模型的训练未采用任何形式的数据增强或正则化操作,以便纯粹评估其建模能力与收敛特性。

为全面评估基准 U-Net 模型在验证集上的性能表现,图~\ref{fig:base_unet_metrics} 展示了训练过程中多个关键评估指标(包括 Dice 系数、Jaccard 指数、Accuracy、Precision、Recall、F1-Score、Specificity 及 Loss)随 epoch 变化的趋势曲线。训练曲线反映了模型收敛过程及其在训练与验证集上的性能差异,可用于分析模型的拟合能力与泛化效果。

\begin{figure}[!htbp]
    \centering
    \includegraphics[width=\textwidth]{fig/base_unet_metrics.pdf}
    \caption{基准U-Net模型训练过程中的性能指标变化趋势}
    \label{fig:base_unet_metrics}
\end{figure}

表 \ref{tab:unet_epoch_compare} 对比了基线 U-Net 在验证集上的三个关键时间点——首次达到 Dice ≥ 0.70 的第 13 轮、全局最优的第 45 轮以及训练末期(最后 10 轮)——对应的主要性能指标。可以看到,模型在第 13 个 epoch 即取得 0.7077 的 Dice值 和 0.5420 的 Jaccard值,说明其收敛速度较快。随后经过 32 轮的渐进优化,Dice 进一步提升到 0.7469(+3.9 pp),Jaccard 则提升到 0.5747(+3.3 pp),而 Precision 几乎保持不变(≈ 0.826),表明网络 在保证低假阳性的同时显著减少漏检。Val-Loss 同期下降约 12 \%,与性能上升趋势一致。

此外,末 10 轮的均值与标准差(右侧一列)显示各指标波动极小(Dice σ ≈ 0.001),证明模型已进入稳定收敛区间且未出现明显过拟合。基于这一观察,本文后续将第 45 轮作为“最优性能”参考点,而第 13 轮则可作为“早期收敛效率”的对照基准,用以衡量不同改进策略在早期与最终阶段的综合效益。

\begin{table}[htbp]
    \centering
    \caption{基线U-Net模型关键Epoch验证集性能指标对比}
    \label{tab:unet_epoch_compare}
    \begin{tabular}{lcccc}
        \toprule
        指标 (val) & Epoch 13 & Epoch 45 (best) & $\Delta$ & 末10轮均值 \\
        \midrule
        Dice        & 0.7077 & \textbf{0.7469} & +3.9 pp   & 0.734 $\pm$ 0.001 \\
        Jaccard     & 0.5420 & \textbf{0.5747} & +3.3 pp   & 0.561 $\pm$ 0.002 \\
        Precision   & \textbf{0.8266} & 0.8262 & $\approx$0 & 0.822 $\pm$ 0.008 \\
        Recall      & 0.6266 & \textbf{0.6537} & +2.7 pp   & 0.645 $\pm$ 0.010 \\
        Val-Loss $\downarrow$ & 0.01312 & \textbf{0.01150} & $-12.4\%$ & 0.0117 $\pm$ 0.0002 \\
        \bottomrule
    \end{tabular}
\end{table}

\subsubsection{消融实验结果及分析}


\subsection{泛化性测试}
%在不同医学图像模态(如CT、MRI)和不同器官分割任务中测试改进方法的泛化能力。
%分析模型在不同任务中的表现,验证其在多种场景下的适用性和鲁棒性。


\subsection{方法局限性讨论}
%研究在不同大小的训练数据集下,改进方法的模型性能变化。
%分析模型在小数据集和大数据集上的表现差异,评估其对数据量的敏感性和鲁棒性。

\section{总结与展望}

\subsection{研究工作总结}
%总结本论文的主要研究成果,包括对U - Net网络结构的分析、改进方法的探索以及实验验证的结果。
本文开展的具体研究工作和取得的相应成果主要体现在以下几个方面:

在原有 U-Net 架构的基础上,在跳转连接路径中引入了关注机制。该模块利用编码器提取的高级语义特征来指导低级特征的选择,从而有效抑制可能通过跳转连接传播的背景噪声。实验结果表明,与基线模型相比,利用注意力机制增强的 U-Net 网络在分割性能方面取得了明显的改进。

同时,为了适应不同成像模式的数据集,建立了标准化的数据预处理工作流程,以确保输入数据在训练前保持适当的动态范围。此外,为了增强模型的泛化能力,还采用了几何数据增强技术,如随机图像旋转和水平或垂直翻转。实验结果表明,这些增强策略的应用大大提高了模型缓解过拟合的能力。

本研究还将 DiceLoss 和 CrossEntropyLoss 相结合,构建了一个混合损失函数,同时考虑了区域级轮廓重叠和像素级分类稳定性。这种方法有助于减轻模型训练过程中因分类不平衡造成的负面影响。实验验证证实,使用这种混合损失函数的模型始终优于使用单一损失函数的模型。

最后,本研究通过系统地进行消融和增强实验,有效地整合了有助于提高性能的各种模块和策略,构建了基于 U-Net 的改进模型AAH U-Net。实验结果表明,AAH U-Net 在几乎所有评价指标上的表现都优于其他增强型模型和策略,展示了强大的泛化能力。


\subsection{研究展望}
%对未来进一步研究的方向进行展望,如探索新的网络架构、结合多模态医学图像进行分割等。

以本研究为基础,未来的工作可以围绕三维建模和弱监督学习展开研究:

一方面,目前基于二维切片的处理策略虽然在计算效率上有优势,但难以充分捕捉 CT/MRI 等三维医学影像的层间空间连续性,导致容积分割结果的局部不一致。引入三维卷积或混合维度建模有望通过轴向关注机制或稀疏卷积优化体素级特征融合,在保持解剖结构拓扑完整性的同时提高微小病灶的空间一致性。

另一方面,面对标注成本高昂的临床现实,半监督和自监督学习策略将成为突破数据瓶颈的关键--例如,利用对比学习构建图像表征的先验知识,或通过图像修复、掩膜重建等代理任务挖掘未标注数据的潜在语义信息,可显著降低对全监督信号依赖的依赖,尤其适用于罕见病病理数据或新兴成像模式的快速适应。

最后,还可以深入探索多模态特征融合机制,例如设计一个跨模态注意力模块,通过动态权重分配自适应地整合不同成像模态的判别特征,从而解决单一模态信息缺失或噪声干扰的问题。此外,利用未标记的多模态数据结合对比学习框架对跨模态语义嵌入空间进行预训练,可以显著提高模型在小样本情况下的泛化能力。
\section{总结与展望}

\subsection{研究工作总结}
%总结本论文的主要研究成果,包括对U - Net网络结构的分析、改进方法的探索以及实验验证的结果。

\subsection{研究创新点总结}
%再次强调本研究的创新点及其在医学图像语义分割领域的价值。


\subsection{研究展望}
%对未来进一步研究的方向进行展望,如探索新的网络架构、结合多模态医学图像进行分割等。

%============= 参考文献及附录 ===============
\bibliographystyle{gbt7714-numerical}          % 使用GB/T 7714的数字型引用格式
\addcontentsline{toc}{section}{参考文献}       % 将“参考文献”添加到目录中,并作为一个新的section
{\zihao{5} \songti \bibliography{bibfile}}     % 设置参考文献的字体为宋体小五号,引用bibfile文件中的文献
\clearpage                                     % 插入分页符,保证参考文献后开始新的一页
{\zihao{5} \songti \bibliography{bibfile}}     % 设置参考文献的字体为宋体小五号,引用bibfile文件中的文献

%=============  致谢  ======================
\section*{致 ~~ 谢}
%===================   结束正文    ===============================================================
\end{document}