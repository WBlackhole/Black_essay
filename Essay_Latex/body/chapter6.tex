\section{总结与展望}

\subsection{研究工作总结}
%总结本论文的主要研究成果,包括对U - Net网络结构的分析、改进方法的探索以及实验验证的结果。
本文所进行的具体研究工作以及所取得的相应成果主要呈现于以下几个方面:

在原有的U-Net架构之上,于跳跃连接路径当中引入了注意力机制,此模块借助编码器提取的高级语义特征来对低级特征的选择给予指导,可有效抑制可能经由跳转连接传播的背景噪声,实验结果显示,相较于基线模型,运用注意力机制提高后的U-Net网络在分割性能方面有了较为十分突出的提升。

为了适配不同成像模式的数据集,本研究建立起了标准化的数据预处理工作流程,以此保证输入数据在训练之前维持适宜的动态范围,并且为了提高模型的泛化能力,还采用了如随机图像旋转以及水平或垂直翻转等几何数据增强技术,实验结果说明,这些提高策略的运用较大提高了模型缓解过拟合的能力。

本研究还把Dice Loss和CrossEntropy Loss结合起来,构建了一个混合损失函数,同时兼顾了区域级轮廓重叠以及像素级分类稳定性,这种方法可减轻模型训练过程中因分类不平衡所造成的负面效应,实验验证说明,使用这种混合损失函数的模型始终比使用单一损失函数的模型表现更优。

最后,本研究借助系统地开展消融和提高实验,有效地整合了可提升性能的各类模块和策略,构建了基于U-Net的改进模型AAH U-Net,实验结果说明,AAH U-Net在几乎所有评价指标上的表现都要优于其他提高型模型和策略,呈现出了强大的泛化能力。


\subsection{研究展望}
%对未来进一步研究的方向进行展望,如探索新的网络架构、结合多模态医学图像进行分割等。

基于本研究,未来工作可围绕三维建模以及弱监督学习展开:

一方面,当下基于二维切片的处理策略虽在计算效率方面有优势,然而却难以充分捕捉 CT/MRI 等三维医学影像的层间空间连续性,致使容积分割结果出现局部不一致的情况,引入三维卷积或者混合维度建模,有望借助轴向关注机制或者稀疏卷积对体素级特征融合给予优化,在维持解剖结构拓扑完整性的同时提升微小病灶的空间一致性。

另一方面,鉴于标注成本高昂这一临床现实,半监督以及自监督学习策略会成为突破数据瓶颈的关键所在——举例来说,借助对比学习构建图像表征的先验知识,或者凭借图像修复、掩膜重建等代理任务挖掘未标注数据的潜在语义信息,可大幅降低对全监督信号的依赖,特别适用于罕见病病理数据或者新兴成像模式的快速适应。还可深入剖析多模态特征融合机制,比如设计一个跨模态注意力模块,借助动态权重分配自适应地整合不同成像模态的判别特征,解决单一模态信息缺失或者噪声干扰的问题,利用未标记的多模态数据结合对比学习框架对跨模态语义嵌入空间进行预训练,可有效提高模型在小样本情形下的泛化能力。