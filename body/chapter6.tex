\section{总结与展望}

\subsection{研究工作总结}
%总结本论文的主要研究成果,包括对U - Net网络结构的分析、改进方法的探索以及实验验证的结果。
本文开展的具体研究工作和取得的相应成果主要体现在以下几个方面:

在原有 U-Net 架构的基础上,在跳转连接路径中引入了关注机制。该模块利用编码器提取的高级语义特征来指导低级特征的选择,从而有效抑制可能通过跳转连接传播的背景噪声。实验结果表明,与基线模型相比,利用注意力机制增强的 U-Net 网络在分割性能方面取得了明显的改进。

同时,为了适应不同成像模式的数据集,建立了标准化的数据预处理工作流程,以确保输入数据在训练前保持适当的动态范围。此外,为了增强模型的泛化能力,还采用了几何数据增强技术,如随机图像旋转和水平或垂直翻转。实验结果表明,这些增强策略的应用大大提高了模型缓解过拟合的能力。

本研究还将 DiceLoss 和 CrossEntropyLoss 相结合,构建了一个混合损失函数,同时考虑了区域级轮廓重叠和像素级分类稳定性。这种方法有助于减轻模型训练过程中因分类不平衡造成的负面影响。实验验证证实,使用这种混合损失函数的模型始终优于使用单一损失函数的模型。

最后,本研究通过系统地进行消融和增强实验,有效地整合了有助于提高性能的各种模块和策略,构建了基于 U-Net 的改进模型AAH U-Net。实验结果表明,AAH U-Net 在几乎所有评价指标上的表现都优于其他增强型模型和策略,展示了强大的泛化能力。


\subsection{研究展望}
%对未来进一步研究的方向进行展望,如探索新的网络架构、结合多模态医学图像进行分割等。

以本研究为基础,未来的工作可以围绕三维建模和弱监督学习展开研究:

一方面,目前基于二维切片的处理策略虽然在计算效率上有优势,但难以充分捕捉 CT/MRI 等三维医学影像的层间空间连续性,导致容积分割结果的局部不一致。引入三维卷积或混合维度建模有望通过轴向关注机制或稀疏卷积优化体素级特征融合,在保持解剖结构拓扑完整性的同时提高微小病灶的空间一致性。

另一方面,面对标注成本高昂的临床现实,半监督和自监督学习策略将成为突破数据瓶颈的关键--例如,利用对比学习构建图像表征的先验知识,或通过图像修复、掩膜重建等代理任务挖掘未标注数据的潜在语义信息,可显著降低对全监督信号依赖的依赖,尤其适用于罕见病病理数据或新兴成像模式的快速适应。

最后,还可以深入探索多模态特征融合机制,例如设计一个跨模态注意力模块,通过动态权重分配自适应地整合不同成像模态的判别特征,从而解决单一模态信息缺失或噪声干扰的问题。此外,利用未标记的多模态数据结合对比学习框架对跨模态语义嵌入空间进行预训练,可以显著提高模型在小样本情况下的泛化能力。