\section{总结与展望}

\subsection{研究工作总结}
%总结本论文的主要研究成果,包括对U - Net网络结构的分析、改进方法的探索以及实验验证的结果。
本论文针对医学图像分割任务中U-Net模型存在的语义间隙大、小目标漏检及跨模态泛化能力不足等问题,从数据预处理、模型架构与训练策略三个维度展开系统性优化,通过基于ISIC 2018数据集的消融增广实验以及基于LiTS 2017和BraTS 2020数据集的泛化性测试,得到了不同模型和策略的实验结果。经过严谨深刻的实验结果数据分析后,证明了本文提出的改进模型AAH U-Net模型的性能优越性。具体的研究工作和成果如下:

\begin{enumerate}
    \item U-Net结构分析与改进:在原网络架构的基础上引入注意力机制,具体来说是在跳跃连接通路在嵌入注意力门模块。通过对编码器的编码器高层语义引导浅层特征筛选,抑制跳跃连接中的背景噪声。实验结果证明,引入注意力机制后的U-Net网络的分割性能较基准模型而言有了显著提升。
    \item 数据处理和增强策略:针对不同成像模态的数据集,本文构建了标准化的预处理流程,保证数据在进行训练前拥有抑制的动态范围。同时,采用了几何增强策略对原始数据进行数据增强,具体来说包括图像随机旋转、水平或垂直翻转。实验结果表明,采取数据增强策略后,模型抑制过拟合的能力有较大提升。
    \item 设计混合损失函数:通过采用DiceLoss和CrossEntropyLoss等权重的混合损失函数,使得网络兼顾全局轮廓重叠度与像素级分类稳定性,缓解类别不平衡问题。实验结果表明,采用混合损失函数的模型比采用单一损失函数的模型性能更优。
    \item 模型的整合和泛化验证:通过对消融增广实验的分析,在融合了不同能提升模型性能的模块和策略后构建了基于U-Net网络的改进模型AAH U-Net。实验结果证明,AAH U-Net几乎所有的评估指标值都优于其他改进模型和策略。
\end{enumerate}

\subsection{研究展望}
%对未来进一步研究的方向进行展望,如探索新的网络架构、结合多模态医学图像进行分割等。

以本研究为基础,对之后的研究进行展望,未来工作可围绕三维建模与弱监督学习深化研究:

一方面,当前基于二维切片的处理策略虽在计算效率上具有优势,却难以充分捕捉CT/MRI等三维医学影像的层间空间连续性,导致体积分割结果存在局部不一致性。引入三维卷积或混合维度建模有望通过轴向注意力机制或稀疏卷积优化体素级特征融合,在保留解剖结构拓扑完整性的同时提升微小病灶的空间一致性。

另一方面,面对标注成本高昂的临床实际,半监督与自监督学习策略将成为突破数据瓶颈的关键——例如,利用对比学习构建影像表征先验知识,或通过图像修复、掩码重建等代理任务挖掘未标注数据的潜在语义信息,可显著降低对全监督信号的依赖,尤其适用于罕见病病理数据或新兴成像模态的快速适配。

最后,可深入探索多模态特征融合机制,例如设计跨模态注意力模块,通过动态权重分配自适应整合不同成像模态的判别性特征,从而解决单一模态信息缺失或噪声干扰问题。此外,结合对比学习框架,利用未标注的多模态数据预训练跨模态语义嵌入空间,可显著提升小样本场景下的模型泛化性。在临床应用层面,此类方法有望推动多模态影像辅助诊断系统的落地,例如通过PET-CT融合精准定位肿瘤代谢活跃区域,或联合超声与MRI实现乳腺癌早期病灶的微钙化点识别。这一方向的突破将不仅提升分割精度,更能为多维度诊疗决策提供可靠依据,助力个性化医疗的发展。