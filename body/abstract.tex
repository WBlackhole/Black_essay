\section*{\zihao{-2} \centering 摘 ~~ 要}

\vskip0.5cm

本文基于注意力机制提出了改进的U-Net模型AAH U-Net,通过在网络中嵌入注意力门,同时采用数据增强和混合损失函数策略,AAH U-Net最终实现了在多个医学图像数据集上的语义分割性能提升。

针对U-Net网络中跳跃连接会将大量来自编码器的背景噪声传递至解码器的问题,本文提出了改进的跳跃连接设计,通过在跳跃连接中嵌入注意力门模块动态地筛选并加权特征图,加强对编码器特征图的特征提取,从而提高模型对目标区域的定位能力。

针对单一损失函数难以适应医学图像中普遍存在的类别不平衡和边界模糊问题,AAH U-Net采用了混合函数策略,通过使用Dice损失函数和交叉熵损失函数的加权混合损失函数,兼顾模型在前景区域重叠度与像素级分类准确率的表现,整体提升模型的分割性能和训练稳定性。

针对医学图像分割任务中常见目标结构复杂且样本规模有限的场景,AAH U-Net模型使用了数据增强策略,通过对图像引入翻转、随机旋转等几何变换,增强模型对空间变换的鲁棒性和小样本学习能力。

凭借注意力门机制、数据增强和混合损失函数策略,AAH U-Net实现在皮肤癌镜像数据集、肝癌CT数据集和脑部肿瘤MRI数据集上语义分割性能的大幅提升。相较于原U-Net网络,AAH U-Net在皮肤癌镜像数据集上的验证集Dice系数从0.75提升至0.83,相对提升约11\%,同时在肝癌CT数据集和脑部肿瘤MRI上的验证集Dice系数分别为0.89和0.83,均大幅优于U-Net模型。

所有代码和训练好的AAH U-Net模型已开源至github:\url{https://github.com/WBlackhole/Black_essay}。



\vskip0.5cm

\addcontentsline{toc}{section}{摘要}
