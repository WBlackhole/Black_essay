\section{基准U-Net模型设计与实现}

\subsection{数据集准备与预处理}
%介绍用于实验的医学图像数据集,包括数据来源、图像类型、标注情况等。
%数据预处理方法,如图像大小调整、归一化处理等。


\subsection{基准U-Net模型构建}
%详细描述U - Net模型的构建过程,包括网络层的搭建、损失函数的选择(如Dice损失函数)等。
%训练过程中的参数设置,如学习率、批量大小、训练轮数等。
%数据增强方法的应用,如旋转、翻转、裁剪等,以增加数据多样性,提高模型的泛化能力。


\subsection{训练策略与参数设置}
%介绍用于评估医学图像语义分割模型性能的指标,如Dice相似系数(Dice Similarity Coefficient,DSC)、Jaccard指数、平均精度均值(mAP)等。
%讲解这些指标的计算方法及其在评估模型分割精度中的作用。
