\pagenumbering{arabic}
\section{绪论}

\subsection{研究背景及意义}
\subsubsection{研究背景}

医学图像是指利用医学成像技术生成的视觉图像,涵盖计算机断层扫描(CT)、磁共振成像(MRI)、超声成像(US)等多种成像模态,通过分析其提供的组织结构、解剖细节及病理信息,放射科医师和内科医师可以快速准确地进行医疗诊断。在分析的过程中,对医学图像进行图像分割是一个关键步骤,它涉及将图像划分为解剖结构或病灶区域(如器官、组织或病变)相对应的不同区域,这种支持肿瘤定位、器官边界划定的分割可以精确解释医学图像,为临床应用中的准确诊断、术前规划和定量分析奠定了基础\cite{panayides2020}。鉴于其重要性和对临床结果的直接影响,医学图像分割的准确性成为现代医疗健康系统工作中不可或缺的要求。然而,当前临床实践中大量依赖放射科医生手动标注分割图像,不仅耗时费力,还存在一定的主观性和一致性问题。尤其在高分辨率三维图像中,手动标注的工作量极大,容易引发标注者疲劳、遗漏或偏差,进而影响诊断质量。因此,伴随着医学图像数量的激增,研究高效、自动化且高精度的医学图像分割方法,不仅能偶减轻医生负担、提升标注一致性,更在促进智能辅助诊疗系统落地中扮演着不可替代的角色。

% 插入图像(医学影像及其手工标注图)

在医学图像分割的发展早期,主要利用如阈值分割、区域生长、边缘检测、聚类分割等传统算法进行组织、器官及病灶区域的识别与提取。这些方法大多依赖人工设计特征或强先验假设,基于像素灰度、边缘梯度、区域一致性等低层视觉信息构建规则,具有较高的可解释性和计算效率。然而,面对实际医学影像数据的复杂性,这些方法暴露出明显的局限性。医学影像中常见的解剖结构形态复杂、边界模糊(如肿瘤与周围组织过渡不清)、组织密度差异微弱,使得依赖简单灰度或梯度判断的传统方法难以准确建模区域间差异。此外,医学影像中的特殊挑战进一步削弱了传统方法的泛化能力与适应性:包括成像过程中引入的噪声干扰、患者间器官形态的个体差异与非线性形变、边界信息的不确定性、以及多模态成像之间强度分布的显著差异\cite{mohdsagheer2020}。这些因素对传统算法构成了严峻挑战,使其在跨患者、跨设备或跨模态应用中难以保持一致性能。同时,在临床实际应用中,常需依赖专业人员进行参数调整、种子点选择或后处理操作,操作流程繁琐、效率低下,严重制约了其在大规模医学影像分析系统中的推广与部署。特别是在高分辨率三维数据或多模态联合分析任务中,传统方法更易出现处理瓶颈,难以满足现代医学精细化、定量化分析的需求。因此,尽管传统图像分割方法在特定条件下仍具参考价值,但其在面对复杂医学图像时的表达能力与适应性显然不足,亟需更具自动学习能力和结构建模能力的先进方法来突破其局限。

近年来,深度学习,尤其是卷积神经网络,在图像分类、检测与分割等计算机视觉任务中取得了突破性进展。其端到端的特征提取能力和强大的表征学习能力,使得图像分割任务由传统手工设计特征的范式,转向自动特征学习与高维特征建模的方向。这一变革也深刻影响了医学图像分割领域。最早将深度学习应用于图像分割的代表性模型包括全卷积神经网络与SegNet等。全卷积神经网络通过去除全连接层,将图像映射为像素级别的类别预测图,是端到端分割网络的开端\cite{shelhamer2016};SegNet在其基础上引入池化索引进行上采样,提升了细节还原能力。然而,这些早期模型在医学图像场景下仍面临一定局限:如对弱边界区域识别不敏感,对结构复杂或形变剧烈的解剖区域分割精度有限,且对小样本训练数据依赖较强,泛化能力不足。

为应对医学图像小样本、边界模糊等挑战,Ronneberger等人\cite{ronneberger2015}于2015年提出了经典的U-Net模型,成为医学图像分割领域的里程碑。U-Net采用编码器-解码器结构,前半部分通过卷积与池化提取高层语义特征,后半部分通过反卷积逐步还原空间分辨率。同时,网络在对称位置引入跳跃连接,将编码器中浅层的高分辨率特征与解码器中对应层的特征图进行拼接融合,从而有效弥补了深层语义特征中局部细节的丢失,实现了多尺度特征的整合与边界定位能力的增强。在ISBI细胞分割挑战赛中,U-Net凭借其出色的结构设计,以显著优势获得冠军。这一成功也推动了U-Net成为后续医学图像分割研究的基准模型,被广泛应用于脑肿瘤、乳腺癌、肺部病灶、眼底血管、皮肤病变等多种应用场景,并衍生出众多变体。U-Net的提出不仅标志着深度学习在医学图像分割领域的广泛应用起点,也为融合传统知识与深度模型提供了范式基础,奠定了现代医学图像分割算法的主流框架。

% 插入图片(图像分割方法的发展图)

尽管U-Net在医学图像分割中取得了巨大的成功,但随着临床场景的复杂化与精度需求的提高,原始U-Net在多种实际应用中仍存在显著局限性。首先,在多目标重叠、边界模糊或形态不规则的复杂场景中,U-Net的分割准确性容易下降。例如,面对多个病灶区域相互接近或重叠(如多发性肿瘤、血管交叉结构)时,模型难以有效区分各目标,易发生融合或漏检现象;对于尺寸较小的病灶(如早期病变、微小转移灶),因其在特征图中易被下采样过程压缩甚至丢失,导致分割结果中频繁出现漏检;此外,U-Net对输入图像质量也较为敏感,在存在伪影、噪声或成像伪差等干扰时,模型的鲁棒性难以保证\cite{azad2024}。

随着医疗影像技术的不断发展和临床需求的日益提升,医学图像分割任务面临着更高层次的挑战与期望。传统的二维静态图像处理已难以满足当前复杂的医学场景,特别是在涉及动态器官(如心脏、肺部)或介入手术导航等实时性强的应用中,分割模型不仅需要具备快速响应能力,还要在保持精度的同时实现高效率推理。此外,随着三维成像技术的普及,诸如CT、MRI等医学影像数据普遍具有体积属性,甚至在某些应用中演化为四维(随时间变化的3D序列)数据,这对模型提出了在空间与时间维度上同时建模的能力要求。因此,支持3D/4D数据结构处理已成为医学图像分割发展的重要方向。

针对上述高阶需求,现有研究虽已在多个方向取得一定进展,但仍存在明显空白,特别是如何针对具体医学应用场景对U-Net进行定制化改进的问题仍缺乏系统探索\cite{krithikaaliasanbudevi2022}。例如,对于多模态影像(如PET-CT、T1/T2-weighted MRI等)带来的信息互补特性,如何在U-Net结构中引入多模态融合机制,以提升模型对异构数据间语义关联的建模能力。此外,在实际临床环境中,分割模型常需部署于移动端、边缘设备或术中设备中运行,受限于算力、内存及功耗等资源约束,因此U-Net结构的轻量化设计亦成为关键研究方向之一,如采用深度可分离卷积、网络剪枝、知识蒸馏等手段减小模型体积和计算量,而不显著损失精度。总体而言,如何在保持U-Net核心优势的基础上,围绕注意力建模、多源信息融合与计算效率优化等维度进行针对性设计,构建适用于多样医学场景的高效、稳健、可解释的新型分割模型,是当前研究中亟需填补的重要空白。

\subsubsection{研究意义}

本研究的学术意义主要体现在三个方面:

从理论价值层面出发,对U-Net模型进行系统性改进,提升其对复杂结构、细粒目标及低对比度区域的建模能力,不仅有助于解决医学图像分割中的多个核心难题,也为设计具备更强表达能力与泛化能力的深度模型提供新的思路。而改进后的模型架构及训练策略在结构设计、信息流建模及优化方法等方面均具有一定的创新性,对医学图像语义分割任务的理论体系形成补充,也为应对小样本条件下的训练、提升模型鲁棒性及解释性等问题提供了具有借鉴价值的解法。

从应用价值层面出发,在实际临床中,医学图像分割是许多关键诊疗环节的基础,尤其在肿瘤检测、器官分割、病灶定量评估等任务中具有核心地位。提高分割模型的精度与效率,可有效辅助医生快速准确地定位病灶区域(如脑肿瘤、乳腺结节、血管斑块等),减少漏诊误判,提升诊断一致性,显著缓解医生的工作负担。特别是在多目标重叠、边界模糊等复杂场景中,基于改进U-Net的高性能模型可为医生提供更清晰、准确的辅助分割结果,提升临床诊疗的整体质量。同时,高质量的自动分割模型可广泛应用于放射治疗中的靶区勾画、术前路径规划、术中导航与术后评估中,减少对经验丰富医生的高度依赖,降低因人工勾画不一致带来的手术误差与放疗剂量偏差,为治疗过程的标准化与精细化提供技术保障。此外,结合患者的个体影像特征与多模态医疗数据,本研究成果也将为实现个性化医疗提供支持,如用于移植器官体积与形状评估、疗效追踪、慢病进展预测等,有助于推动以患者为中心的精准医疗落地。

从社会价值层面出发,面向基层医疗或资源匮乏地区,具备轻量化、高精度、可部署性的医学图像分割模型,可构建低成本的智能辅助诊断系统,缓解医资不均、专家短缺等公共卫生难题,推动普惠医疗的发展。因此,本研究不仅具备显著的学术理论价值,也具有广泛而深远的临床实践与社会应用前景。


\subsection{国内外研究现状}

自2015年Ronneberger等人提出U-Net以来,该网络凭借其高效的特征提取与恢复能力,在医学影像的任务中取得了广泛的应用。然而,传统U-Net在全局信息建模和边界细化方面仍存在一定不足,因此近年来的研究重点逐渐转向U-Net的结构优化、注意力机制的引入以及多模态数据融合。针对不同的应用场景,国内外学者提出了多种改进方案。

\subsubsection{国外研究现状}

在国际上,U-Net的改进主要集中在以下几个方向\cite{azad2024}:

\begin{enumerate}
    \item 多尺度特征融合:Milletari等人\cite{milletari2016}设计了V-Net,将3D卷积引入U-Net框架,以适应三维医学影像的分割任务。此外,zhou等人\cite{zhou2018}提出了Unet++,通过密集跳跃连接优化特征重用,实现了更精细的多尺度特征融合,改善了小目标与弱边界的分割性能。
    \item 引入注意力机制:
    \item 混合架构:
    \item 轻量化设计:
\end{enumerate}

可以发现,国外医学图像分割研究在算法的创新性、数据资源丰富度及跨学科合作的紧密性等方面表现出明显优势。研究者尝试将Transformer、注意力机制、多模态融合等最新技术融入UNet结构中,不断突破传统模型的局限。此外,国际主导的开源数据集(如脑肿瘤分割BraTS、肝脏分割LiTS)和跨学科竞赛平台的丰富资源也有效推动了技术发展与应用落地。

然而,国外研究在某些复杂场景下仍存在明显不足,例如对多器官重叠、小病灶等困难场景的分割仍具挑战性,且现有模型在临床应用中的可解释性不足,难以清晰展示决策依据,这成为未来医学图像分割领域的重要发展方向。

总体而言,国外的研究持续深入推动医学图像分割领域的发展,不断丰富方法论与实践经验,为医学人工智能的发展奠定了坚实基础。

\subsubsection{国内研究现状}

\subsubsection{研究现状总结}

\subsection{研究内容与创新点}
%明确本论文的研究内容,包括对U - Net网络结构的分析、改进方法的探索以及实验验证等。
%突出本研究的创新点,如对网络结构的优化、新数据增强方法的引入等。


\subsection{论文组织结构}
%简要介绍各章节的主要内容和逻辑关系。
